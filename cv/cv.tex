\documentclass[a4paper]{article}

\usepackage[french]{babel}
\usepackage[utf8]{inputenc}
\usepackage[T1]{fontenc}

\usepackage{latexsym}
\usepackage[empty]{fullpage}
\usepackage{titlesec}
\usepackage{marvosym}
\usepackage[usenames, dvipsnames]{color}
\usepackage{verbatim}
\usepackage{enumitem}
\usepackage[pdftex]{hyperref}
\usepackage{fancyhdr}

% page style
\pagestyle{fancy}
\fancyhf{} % clear all header and footer fields
\fancyfoot{}
\renewcommand{\headrulewidth}{0pt}
\renewcommand{\footrulewidth}{0pt}

% Adjust margins
\addtolength{\oddsidemargin}{-0.530in}
\addtolength{\evensidemargin}{-0.375in}
\addtolength{\textwidth}{1in}
\addtolength{\topmargin}{-.45in}
\addtolength{\textheight}{1in}

\urlstyle{rm}

\raggedbottom
\raggedright
\setlength{\tabcolsep}{0in}

% Sections formatting
\titleformat{\section}{
  \vspace{-10pt}\scshape\raggedright\large
}{}{0em}{}[\color{black}\titlerule 
% \vspace{-6pt}
]

% cv commands
\newcommand{\cvitem}[2]{
	\item{
		\textbf{#1}{ : #2 } 
		%\vspace{-2pt}
	}
}
\newcommand{\cvsubitem}[2] {\cvitem{#1}{#2} 
%\vspace{-3pt}
}

\newcommand{\cvsubheading}[4]{
  \vspace{-1pt}\item
    \begin{tabular*}{0.97\textwidth}{l@{\extracolsep{\fill}}r}
      \textbf{#1} & #2 \\
      \textit{#3} & \textit{#4} \\
    \end{tabular*}
%\vspace{-5pt}
}

\begin{document}

% Heading
\begin{tabular*}{\textwidth}{l@{\extracolsep{\fill}}r}
	\textbf{{\LARGE Alexis Hoffmann}} & Email : \href{mailto:}{hoffmann.itconsulting@gmail.com}\\
	\href{https://github.com/R3B3-888}{Github : github.com/R3B3-888} & \href{https://alexis-hoffmann.emi.u-bordeaux.fr}{Portofolio: alexis-hoffmann.emi.u-bordeaux.fr}\\
	\href{https://gitlab.com/R3B3-888}{Gitlab : gitlab.com/R3B3-888} & \href{https://www.linkedin.com/in/alexis-hoffmann-b026331b7/}{Linkedin : linkedin.com/in/alexis-hoffmann}
\end{tabular*}

% Education
\section{~~Formations}
	\begin{itemize}[leftmargin=*]
		\cvsubheading{Master en Informatique}{Université de Bordeaux}{Section ASPIC (Autonomous Systems, Perception, Interaction and Control)}{Septembre 2020 - Septembre 2022}
		\cvsubheading{Licence en Informatique}{Université de Bordeaux}{Obtention du C2I durant la Licence}{Septembre 2017 - Mai 2020}
		\cvsubheading{Baccalauréat}{François Mauriac - Bordeaux}{Section Scientifique mention Bien}{2017}
	\end{itemize}

\section{Sommaire des compétences}
	\begin{itemize}[leftmargin=*]
		\cvsubitem{Langages}{~~~~~~~Python, C, C\#, Java, Bash/Shell, SQL, AutoLisp, {\LaTeX}, PHP}
		\cvsubitem{Frameworks}{~~~Scikit-Learn, TensorFlow, Keras, Qt, OpenCV}
		\cvsubitem{Outils}{~~~~~~~~~~~~~Anaconda, GIT, Gitlab, Ganttproject, Trello, AutoCAD map, Unity, Vim}
		\cvsubitem{Plateformes}{~~~~Linux, Windows, MacOS, Arduino, Raspberry pi}
		\cvsubitem{Humaines}{~~~~~~~Agilité, Organisation, Autonomie, Gestion du temps, Esprit d'équipe}
		\cvsubitem{Langues}{~~~~~~~~~~Anglais courant, Chinois et Arabe notions}
	\end{itemize}

\section{Expériences}
	\begin{itemize}[leftmargin=*]
		\cvsubheading{Stage au LaBRI pour l'association BeesForLife}{Bordeaux}
{Détection de nid de frelons asiatiques (temps complet)}{Avril 2022 - Présent}
		\begin{itemize}
			\cvitem{Technologies utilisés}{Python, Anaconda, Tensorflow, Qt}
			\cvitem{Interopérabilité de l'application}{Client fonctionnant avec un ordinateur sous MacOS}
			\cvitem{Dataset unique fourni par le client}{Images couleurs et infra-rouges capturées depuis des drones}
			\cvitem{Détection à partir d'un modèle CNN}{Mask RCNN et méthode d'élimination de faux positifs}
			\cvitem{Recherche d'un modèle de l'état de l'art}{Vision Transformer et Vidéo Vision Transformer en développement}
		\end{itemize}
		\cvsubheading{Formateur sur AutoCAD avec scripting AutoLISP}{AtiCube}{Formation sur l'utilisation des scripts AutoLISP}{Février 2022}
		\cvsubheading{Tuteur Linux et bash pour élèves malvoyants}
{Cellule PHASE à l'Univ. Bordeaux}{Cours de bash avancé pour niveau Licence 3 Pro}{Octobre 2020 - Février 2022}
		\cvsubheading{Stage à APAVE SUDOUEST SAS}
{Artigues-Près-Bordeaux - 33370}{Manipulation de base de données - SQL, support client nv. 3}{Mai 2019}
	\end{itemize}

\section{Projets}
	\begin{itemize}[leftmargin=*]
		\cvsubitem{Création d'une application pour manipuler des cadastres}
{(En cours de developpement) Webscrapping avec api de cadastres.gouv.fr. Objectif de l'appli : conversion automatique de fichiers DXF en DWG sur autant de cadastres voulu. Tech. : Python, Pyinstaller, ODA, PyQt6, AutoCAD Map}
		\cvsubitem{Essaim de drone résilient effectuant de la surveillance de plage}{POC sur Unity avec développement TDD (Février-Mars 2022). Tech. : C\#, Kanban}
		\cvsubitem{Essaim de drone en boid}{Implémentation de l'algorithme de boid. Tech. : Unity, C\#}
		\cvsubitem{Programmation AVR}{Horloge sur des leds fixées sur des pales de ventilateur en vue persistante. Tech. : C, µcontrôleur atmega328p}
		%\cvsubitem{Drone terrestre suiveur de ligne}{Odométrie, détection de ligne à partir d'un algorithme de reconnaissance de forme et manipulation de servomoteurs}
		%\cvsubitem{Implémentation du framework EasyPAP}{Jeu life a été codé en MPI, AVX et OpenGL dans le cadre d'une UE de programmation parallèle}
		\cvsubitem{Création d'un simulateur de Robocub}{Architecture modulaire pour entraîner des modèles d'apprentissage par renforcement. Tech. : Python}
		\cvsubitem{Recréation d'AlphaGo}{Construction et entraînement du modèle pour un tournoi de 57 autres AlphaGo, fini 3e. Tech. : Keras}
		%\cvsubitem{Développement d'un site marchand}{POC pour mettre en place avec Angular un premier site utilisable pour la gestion de paniers}
		%\cvsubitem{Développement d'une application web pour la gestion de matelas de contention pour le service radiologie au CHU de Bordeaux}{Construit avec Symfony en équipe de 3 personnes}
	\end{itemize}


\section{Loisirs et centres d'intérêt}

Tennis, Surf, Ski, Catamaran, cinéma, musculation


\end{document}
